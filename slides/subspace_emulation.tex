%!TEX program = lualatex

\documentclass[xcolor=dvipsnames, aspectratio=169]{beamer}
\usetheme[
    outer/progressbar=foot,
    ]{metropolis}           % Use metropolis theme
% \usecolortheme[snowy]{owl}


% \usepackage{footbib}

\usepackage{graphicx,amsmath,amssymb,bm}
\usepackage{listings}
\usepackage{xcolor}
\usepackage{tikz, pgfplots, filecontents, tikzscale}
\usetikzlibrary{arrows, arrows.meta, tikzmark}
\usetikzlibrary{shapes.geometric, shapes}
\usetikzlibrary{calc}
\tikzstyle{every picture}+=[remember picture]
\usepackage{pgfpages}
\usepackage{multimedia}
\usepackage{media9}
\usepackage{tabu}
\usepackage{multirow}
\usepackage{standalone}
\usepackage{physics}
\usepackage[most]{tcolorbox}
\usepackage{appendixnumberbeamer}
\tcbuselibrary{listings}
\tcbuselibrary{raster}
\tcbuselibrary{skins}

\usepackage{chngpage}  % Center columns with adjustwidth
\usepackage{mathtools}

\usepackage{booktabs}
\usepackage{emoji}
% \usepackage{minted}

\usepackage{dsfont}


% One style for all TikZ pictures for working with overlays:
\tikzset{every picture/.style=remember picture}
% Define a TikZ node for math content:
\newcommand{\mathnode}[2]{%
   \mathord{\tikz[baseline=(#1.base), inner sep = 0pt]{\node (#1) {$#2$};}}}

\usepackage{natbib}

\newcommand\myfootcite[1]{\footnote{\cite{#1}}}

% Math column
\newcolumntype{M}[1]{>{$}#1<{$}@{\hskip 0.05in}}

\tcbset{
    metropolisstyle/.style={
        enhanced,
        colback=normal text.bg!90!normal text.fg,
        colframe=normal text.fg,
        fonttitle=\bfseries,
        coltitle=normal text.bg,
        colbacktitle=normal text.fg!95!normal text.bg,
        attach boxed title to top center={yshift=-0.25mm-\tcboxedtitleheight/2,yshifttext=2mm-\tcboxedtitleheight/2},
        boxed title style={
            boxrule=0.5mm,
            frame code={ \path[tcb fill frame] ([xshift=-4mm]frame.west)
            -- (frame.north west) -- (frame.north east) -- ([xshift=4mm]frame.east)
            -- (frame.south east) -- (frame.south west) -- cycle; },
            interior code={ \path[tcb fill interior] ([xshift=-2mm]interior.west)
            -- (interior.north west) -- (interior.north east)
            -- ([xshift=2mm]interior.east) -- (interior.south east) -- (interior.south west)
            -- cycle;}
        }
    }
}



\tcbset{
    nostyle/.style={
        enhanced,
        colback=normal text.bg,
        colframe=normal text.bg,
        fonttitle=\bfseries,
        coltitle=normal text.bg,
        colbacktitle=normal text.fg!95!normal text.bg,
        attach boxed title to top center={yshift=-0.25mm-\tcboxedtitleheight/2,yshifttext=2mm-\tcboxedtitleheight/2},
        boxed title style={
            boxrule=0.5mm,
            frame code={ \path[tcb fill frame] ([xshift=-4mm]frame.west)
            -- (frame.north west) -- (frame.north east) -- ([xshift=4mm]frame.east)
            -- (frame.south east) -- (frame.south west) -- cycle; },
            interior code={ \path[tcb fill interior] ([xshift=-2mm]interior.west)
            -- (interior.north west) -- (interior.north east)
            -- ([xshift=2mm]interior.east) -- (interior.south east) -- (interior.south west)
            -- cycle;}
        }
    }
}

\tcbset{
    subbox/.style={
        sidebyside, notitle, nobeforeafter, frame hidden,
        size=minimal, bottom=2mm, top=2mm, valign=center,
        sidebyside gap=4mm
    }
}

\setbeamercolor{background canvas}{bg=white}

% \tcbsubskin{mycross}{empty}{
%           metropolisstyle,
%           frame code={%
%           \draw[red,line width=5pt] (frame.south west)--(frame.north east);
%           \draw[red,line width=5pt] (frame.north west)--(frame.south east);
%           },
%           skin first=mycross,skin middle=mycross,skin last=mycross }

\usetikzlibrary{shapes.geometric, arrows, positioning}

% \newtcolorbox{myblock}[2][]{
%     enhanced,
%     colback=normal text.bg!90!normal text.fg,
%     colframe=normal text.fg,
%     coltitle=normal text.bg,
%     % colbacktitle=blue!5!yellow!10!white,
%     colbacktitle=normal text.fg!95!normal text.bg,
%     attach boxed title to top center={yshift=-0.25mm-\tcboxedtitleheight/2,yshifttext=2mm-\tcboxedtitleheight/2},
%     title=\textbf{#2},
%     boxed title style={
%         boxrule=0.5mm,
%         frame code={ \path[tcb fill frame] ([xshift=-4mm]frame.west)
%         -- (frame.north west) -- (frame.north east) -- ([xshift=4mm]frame.east)
%         -- (frame.south east) -- (frame.south west) -- cycle; },
%         interior code={ \path[tcb fill interior] ([xshift=-2mm]interior.west)
%         -- (interior.north west) -- (interior.north east)
%         -- ([xshift=2mm]interior.east) -- (interior.south east) -- (interior.south west)
%         -- cycle;}
%   },
%   #1
% }

\newtcolorbox{myblock}[2][]{
    metropolisstyle,
    title=#2,
    #1
}




% Renew quote
\usepackage{xparse}

\let\oldquote\quote
\let\endoldquote\endquote

\RenewDocumentEnvironment{quote}{o}
  {\oldquote}
  {\par\nobreak\smallskip
   \hfill\IfValueT{#1}{---~#1}\endoldquote 
   \addvspace{\bigskipamount}}


% ======================================================================
% Begin Theme Styling
% ======================================================================

% Official OSU scarlet color
\definecolor{OSUscarlet}{RGB}{187, 0, 0}

% Tableau 10 palette
% \definecolor{t10blue}{HTML}{1f77b4}
% \definecolor{t10orange}{HTML}{ff7f0e}
\definecolor{t10green}{HTML}{2ca02c}
\definecolor{t10red}{HTML}{d62728}
\definecolor{t10purple}{HTML}{9467bd}
\definecolor{t10blue}{RGB}{31,119,180}
\definecolor{t10orange}{RGB}{255,127,14}
  
% Theme colors are derived from these two elements
% \setbeamercolor{normal text}{%
%     fg=mDarkTeal,
%     bg=black!2
%   }

% Change color bar
% \setbeamercolor{alerted text}{fg=OSUscarlet}

% ... however you can of course override styles of all elements
% \setbeamercolor{frametitle}{bg=OSUscarlet}


% Leave normal text alone, but change primary palette
% \setbeamercolor{palette primary}{%
%   use=normal text,
%   fg=normal text.bg,
%   % bg=normal text.fg,
%   bg=OSUscarlet  % Controls color of standout frames
% }

% ======================================================================
% End Theme Styling
% ======================================================================

% ======================================================================
% Begin Code Styling
% ======================================================================

\definecolor{mygreen}{rgb}{0,0.6,0}
\definecolor{mygray}{rgb}{0.5,0.5,0.5}
\definecolor{mymauve}{rgb}{0.58,0,0.82}
\colorlet{numbercolor}{red!20!violet}


\newtoggle{InString}{}% Keep track of if we are within a string
\togglefalse{InString}% Assume not initally in string

% \newcommand*{\FormatDigit}[1]{\textcolor{numbercolor}{#1}}
\newcommand*{\FormatDigit}[1]{\iftoggle{InString}{#1}{\color{numbercolor}#1}}%
\newcommand*{\ProcessQuote}[1]{\color{mygreen}#1\iftoggle{InString}{\global\togglefalse{InString}}{\global\toggletrue{InString}}}%

\lstset{ %
  % backgroundcolor=\color{white},    % choose the background color
  basicstyle=\footnotesize\ttfamily,  % size of fonts used for the code + typewriter (mono)
  breaklines=true,                    % automatic line breaking only at whitespace
  captionpos=b,                       % sets the caption-position to bottom
  commentstyle=\color{mygray},        % comment style
  escapeinside={-*}{*-},              % if you want to add LaTeX within your code
  keywordstyle=\color{blue},          % keyword style
  stringstyle=\color{mygreen},        % string literal style
  % literate=%
  %  *{0}{{{\color{numbercolor}0}}}1
  %   {1}{{{\color{numbercolor}1}}}1
  %   {2}{{{\color{numbercolor}2}}}1
  %   {3}{{{\color{numbercolor}3}}}1
  %   {4}{{{\color{numbercolor}4}}}1
  %   {5}{{{\color{numbercolor}5}}}1
  %   {6}{{{\color{numbercolor}6}}}1
  %   {7}{{{\color{numbercolor}7}}}1
  %   {8}{{{\color{numbercolor}8}}}1
  %   {9}{{{\color{numbercolor}9}}}1
  literate=%
    % {"}{{{\ProcessQuote{"}}}}1% Disable coloring within double quotes
    % {'}{{{\ProcessQuote{'}}}}1% Disable coloring within single quote
    *{0}{{\FormatDigit{0}}}{1}%
    {1}{{\FormatDigit{1}}}{1}%
    {2}{{\FormatDigit{2}}}{1}%
    {3}{{\FormatDigit{3}}}{1}%
    {4}{{\FormatDigit{4}}}{1}%
    {5}{{\FormatDigit{5}}}{1}%
    {6}{{\FormatDigit{6}}}{1}%
    {7}{{\FormatDigit{7}}}{1}%
    {8}{{\FormatDigit{8}}}{1}%
    {9}{{\FormatDigit{9}}}{1}%
    {.0}{{\FormatDigit{.0}}}{2}% Following is to ensure that only periods
    {.1}{{\FormatDigit{.1}}}{2}% followed by a digit are changed.
    {.2}{{\FormatDigit{.2}}}{2}%
    {.3}{{\FormatDigit{.3}}}{2}%
    {.4}{{\FormatDigit{.4}}}{2}%
    {.5}{{\FormatDigit{.5}}}{2}%
    {.6}{{\FormatDigit{.6}}}{2}%
    {.7}{{\FormatDigit{.7}}}{2}%
    {.8}{{\FormatDigit{.8}}}{2}%
    {.9}{{\FormatDigit{.9}}}{2}%
    %{,}{{\FormatDigit{,}}{1}% depends if you want the "," in color
    {\ }{{ }}{1}% handle the space
}


% \newtcblisting{python}[1][]{
%   listing only,
%   breakable,
%   colframe=gray,
%   colback=gray!10,
%   listing options={
%     language=python,
%     basicstyle=\footnotesize\ttfamily,
%     breaklines=true,
%     columns=fullflexible,
%     morekeywords={with,as},
%     texcl  % Comments are LaTeX
%   },
%   #1
% }

\newtcblisting{python}[1][]{
  listing only,
  % breakable,
  % colframe=gray,
  % colback=gray!10,
  colback=normal text.bg!90!normal text.fg,
  colframe=normal text.fg,
  bottom=-1mm, top=-1mm,
  left=2mm, right=2mm,
  listing options={
    language=python,
    basicstyle=\footnotesize\ttfamily,
    breaklines=true,
    columns=fullflexible,
    morekeywords={with,as,None,True,False,self},
    texcl  % Comments are LaTeX
  },
  #1
}

% Example usage:
% \begin{frame}[fragile]
% \begin{python}[]
% # A comment with \LaTeX: $x^2 = \pi$
% g = 'hello'
% for i in range(10):
%     print(i)
% \end{python}
% \end{frame}

% ======================================================================
% End Code Styling
% ======================================================================
% !TEX root = CH_to_EKM_revisited_v4.tex

% Define \newsubcommand:
% look whether the following character is _ and, if so, it will merge the subscripts.
\makeatletter
\newcommand\newsubcommand[3]{\newcommand#1{#2\sc@sub{#3}}}
\def\sc@sub#1{\def\sc@thesub{#1}\@ifnextchar_{\sc@mergesubs}{_{\sc@thesub}}}
\def\sc@mergesubs_#1{_{\sc@thesub#1}}
\makeatother


\DeclareDocumentCommand{\includevcenteredgraphics}{ O{} m }{%
    \vcenter{\hbox{\includegraphics[#1]{#2}}}
}

\newcommand{\subspace}[1]{ \ensuremath{\widetilde{#1}} }
\newcommand{\normmat}{\mathcal{N}}

\newcommand{\lec}{a}
\newcommand{\lecvec}{\vec{\lec}}
\newcommand{\lecs}{\vec{\lec}}
\newcommand{\Emax}{E_\textup{max}}
\newcommand{\kmax}{k_\textup{max}}

%\newcommand{\NNLO}{N$^2$LO}
%\newcommand{\NNNLO}{N$^3$LO}
\newcommand{\LO}{\ensuremath{{\rm LO}}}
\newcommand{\NLO}{\ensuremath{{\rm NLO}}}
\newcommand{\NNLO}{\ensuremath{{\rm N}{}^2{\rm LO}}}
\newcommand{\NNNLO}{\ensuremath{{\rm N}{}^3{\rm LO}}}
\newcommand{\NNNNLO}{\ensuremath{{\rm N}{}^4{\rm LO}}}
\newcommand{\NkLO}[1]{\ensuremath{\mathrm{N}^{#1}\mathrm{LO}}}

\DeclareMathOperator{\E}{\mathbb{E}}
\DeclareMathOperator{\Var}{\mathbb{V}ar}


\newcommand{\qgeo}[1]{q_k^{(#1)}}
\newcommand{\neff}{\ensuremath{N_{\text{eff}}}}
\newcommand{\data}{\mathcal{D}}
% \newcommand{\hyperparam}{\boldsymbol{\theta}}
\newcommand{\param}{\boldsymbol{\theta}}
\newcommand{\len}{\ell}
\newcommand{\lenvec}{\boldsymbol{\ell}}
\newcommand{\lenset}{\ensuremath{L}}
\newcommand{\lenvecset}{\ensuremath{\mathbf{L}}}
\DeclareMathOperator{\GP}{\mathcal{GP}}
\DeclareMathOperator{\TP}{\mathcal{TP}}

% Neutron-proton
\newcommand{\npr}{\ensuremath{np}}

\newcommand{\chiEFT}{\ensuremath{\chi}\text{EFT}}

\newcommand{\boldvec}[1]{\bm{#1}}
%\newcommand{\boldvec}[1]{\mathbf{#1}}


\newcommand{\abar}{\bar a}

\newsubcommand{\ckvec}{\mathbf{c}}{k}
% \newcommand{\ckvecsq}{\mathbf{c}_k^{2}}
\newsubcommand{\bkvec}{\mathbf{b}}{k}
% \newcommand{\bkvecsq}{\mathbf{b}_k^{2}}
% \newcommand{\kinparvec}{\boldvec{\alpha}}
\newcommand{\kinparvec}{x}
\newcommand{\kinparvecset}{\vb{X}}
\newcommand{\kinparvecmag}{x}
\newsubcommand{\ckvecset}{\mathbf{C}}{k}
\newcommand{\cbarset}{\bar{C}}
\newcommand{\cset}{C}
\newcommand{\capprox}{\ensuremath{c'}}
\newcommand{\capproxset}{\ensuremath{C'}}
\newsubcommand{\ckvecapprox}{\mathbf{c}'}{k}
\newsubcommand{\ckvecapproxset}{\mathbf{C}'}{k}
\newcommand{\bset}{B}
\newcommand{\bapprox}{\ensuremath{b'}}
\newcommand{\bapproxset}{\ensuremath{B'}}
\newsubcommand{\bkvecapprox}{\mathbf{b}'}{k}
\newsubcommand{\bkvecset}{\mathbf{B}}{k}
\newsubcommand{\bkvecapproxset}{\mathbf{B}'}{k}
\newcommand{\genobs}{y}
\newsubcommand{\genobsth}{y}{\textup{th}}
\newsubcommand{\genobsexp}{y}{\textup{exp}}
\newsubcommand{\genobsvec}{\boldsymbol{y}}{\!k}

\newsubcommand{\akvec}{\mathbf{a}}{k}
\newcommand{\aset}{A}
\newcommand{\aapprox}{\ensuremath{a'}}
\newcommand{\aapproxset}{\ensuremath{A'}}
\newsubcommand{\akvecapprox}{\mathbf{a}'}{k}
\newsubcommand{\akvecset}{\mathbf{A}}{k}
\newsubcommand{\akvecapproxset}{\mathbf{A}'}{k}


\newcommand{\Covfunc}{\ensuremath{k}}
\newcommand{\Covmat}{\vb{K}}
\newcommand{\Corrmat}{\vb{R}}
\newcommand{\Covmattest}{\boldsymbol{\Sigma}}


%%%%%%%%%%%%%%%%%%%%%%%%

%%%%%%%%%%%%%%%%%%%
% The \pr command
%%%%%%%%%%%%%%%%%%%

\let\Pr\undefined{}  % Remove the definition from the Physics package

% \newcommand{\pr}{\text{pr}} % Old
\DeclareMathOperator{\pr}{pr} % Good, but want to handle sizing and | spacing?
% \DeclareDocumentCommand\pr{}{\opbraces{\prob}} % Using opbraces from physics package to resize delimeters
\newcommand{\given}{\,|\,}  % Use for | in \pr
\DeclareMathOperator{\prior}{\pi}


\newcommand{\cbar}{\bar{c}}


\newcommand{\Q}{\ensuremath{Q}}
\newcommand{\Qapprox}{\ensuremath{Q'}}
\newcommand{\Qmax}{\Q_{\rm max}}
\newcommand{\smallscale}{\ensuremath{\mathcal{Q}}}
\newcommand{\smallscaleapprox}{\ensuremath{\mathcal{Q}'}}

\newcommand{\Elab}{E_{\rm lab}}


\newcommand{\normal}{\mathcal{N}}
\newcommand{\IG}{\textup{IG}}


\newcommand{\transpose}[1]{#1^{\intercal}}
% \newcommand{\trans}{\intercal}



\newcommand{\genobsref}{\ensuremath{y_{\textup{ref}}}}
\newcommand{\Xref}{\genobsref}


% Differential taken from Physics package
% Uses smart spacing for a nice look. Requires xparse.
% https://www.ctan.org/pkg/physics?lang=en

% First, the basics:
\def\diffd{\mathrm{d}}  % Upright differentials
% \def\diffd{d}  % Italic differentials

% Now add spacing:
% Derivatives
\DeclareDocumentCommand\differential{ o g d() }{ % Differential 'd'
    % o: optional n for nth differential
    % g: optional argument for readability and to control spacing
    % d: long-form as in d(cos x)
    \IfNoValueTF{#2}{
        \IfNoValueTF{#3}
            {\diffd\IfNoValueTF{#1}{}{^{#1}}}
            {\mathinner{\diffd\IfNoValueTF{#1}{}{^{#1}}\argopen(#3\argclose)}}
        }
        {\mathinner{\diffd\IfNoValueTF{#1}{}{^{#1}}#2} \IfNoValueTF{#3}{}{(#3)}}
    }
\DeclareDocumentCommand\dd{}{\differential} % Shorthand for \differential


% Path Derivative:
\newcommand{\pathd}{\mathcal{D}}  % differential symbol for path integrals

% Now add relevant spacing and options
% Inspired by differential definition from `Physics' package at
% https://www.ctan.org/tex-archive/macros/latex/contrib/physics?lang=en
\DeclareDocumentCommand\pathdifferential{ o g d() }{ % Path 'D'
    % o: optional n for nth differential
    % g: optional argument for readability and to control spacing
    % d: long-form as in d(cos x)
    \IfNoValueTF{#2}{
        \IfNoValueTF{#3}
            {\pathd\IfNoValueTF{#1}{}{^{#1}}}
            {\mathinner{\pathd\IfNoValueTF{#1}{}{^{#1}}\argopen(#3\argclose)}}
        }
        {\mathinner{\pathd\IfNoValueTF{#1}{}{^{#1}}#2} \IfNoValueTF{#3}{}{(#3)}}
    }

\DeclareDocumentCommand\DD{}{\pathdifferential} % Shorthand for \pathdifferential

% Observable notation

\makeatletter
\def\dsigma{\@ifstar\@@dsigmawithstar\@dsigma}
\def\@dsigma{\frac{\dd\sigma}{\dd\Omega}}  % \dsigma for upright frac
\def\@@dsigmawithstar{\dd\sigma/\dd\Omega} % \dsigma* for inline frac
\makeatother

\graphicspath{{figures/}{animations/}}

% ======================================================================
% ======================================================================
% Begin Slides
% ======================================================================
% ======================================================================

\title{Variational principles \& efficient subspace emulation}
\date{July 12, 2021}
\author[shortname]{\textcolor{t10orange}{Jordan Melendez}\inst{1,}\,\inst{2}
}
\institute[shortinst]{
  \inst{1} Root Insurance, Data Scientist \and
  \inst{2} The Ohio State University
  }
\begin{document}

\begin{frame}
    \titlepage
    \vspace{-2.1in}
    \begin{columns}
    \begin{column}{0.4\textwidth}
    \end{column}
    \begin{column}{0.5\textwidth}
        \centering
        \includegraphics[width=0.6\textwidth]{root_logo_black.png} \\
        \vspace{0.1in}
        \includegraphics[width=\textwidth]{OSU_EPS_logo-eps-converted-to}
    \end{column}
    \end{columns}
\end{frame}

\begin{frame}

\begin{center}
There is \alert{code} to accompany these slides!

\alert{\href{https://github.com/jordan-melendez/quantum-emulator-examples}{github.com/jordan-melendez/quantum-emulator-examples}}
\end{center}

\end{frame}

\section{Background}

\begin{frame}{Progress in heavy nuclei}

\begin{columns}
\begin{column}{0.4\textwidth}
\begin{itemize}
\item Great progress has been made
\item Growth in computing power and algorithms is pushing to heavier systems
\item But statistics requires more than one prediction
\end{itemize}
\end{column}
\begin{column}{0.6\textwidth}
\begin{figure}
\includegraphics[width=\textwidth]{figures/nuclear_chart_progress_4.png}
\end{figure}
\end{column}
\end{columns}
\end{frame}

\begin{frame}{Why we need emulators}

\begin{columns}
\begin{column}{0.5\textwidth}
Complexity:
\begin{enumerate}
\item \emoji{thinking-face} Forward UQ
\item \emoji{cold-sweat} Inverse UQ
\item \emoji{scream} Experimental Design
\end{enumerate}
\end{column}
\begin{column}{0.5\textwidth}
\begin{itemize}
    \item $r$-process, $0\nu\beta\beta$ simulations, \& optimizing FRIB experiments can each be compute intensive --- but important!
    \item Emulator: An algorithm capable of accurately approximating the exact solution while requiring only a fraction of the computational resources
\end{itemize}
\end{column}

\end{columns}
 
\end{frame}


\begin{frame}[t]{Relation to BAND}

\begin{columns}
\begin{column}{0.4\textwidth}
\begin{myblock}[valign=center]{BAND}
The goal of BAND is to translate novel statistical methods of UQ into
software tools that address prominent current problems in nuclear physics.
\end{myblock}
\begin{itemize}
\item Subspace emulation could play a key role in \alert{Tool A}.
\item An emulator can feed into all subsequent tools
\end{itemize}
\end{column}
\begin{column}{0.6\textwidth}
\begin{figure}
\includegraphics[width=1.05\textwidth]{figures/band_flowchart.png}
\end{figure}
\end{column}
\end{columns}
\end{frame}


\section{Emulators}


\begin{frame}[plain, fragile, t]{Gaussian processes as emulators}

\begin{columns}[t]
\begin{column}{0.6\textwidth}
\begin{python}[]
import numpy as np
from sklearn.gaussian_process \
    import GaussianProcessRegressor
from sklearn.gaussian_process.kernels \
    import RBF, ConstantKernel as C

x_train = np.arange(0, 1.1, 0.2)
x_valid = np.linspace(0, 1, 101)
y_train = np.sin(2 * np.pi * x_train)
y_valid = np.sin(2 * np.pi * x_valid)

kernel = C(1) * RBF(length_scale=0.2)
gp = GaussianProcessRegressor(kernel)
gp.fit(x_train[:, None], y_train)
y_pred, y_stdv = gp.predict(
    x_valid[:, None], return_std=True)
\end{python}
\end{column}
\begin{column}{0.5\textwidth}
\begin{itemize}
\item Forward Problem: Train at ``kinematic'' points
\item Inverse Problem: + train at parameter values
\item Modern libraries make it easy
\end{itemize}
\vspace{-0.3cm}
\begin{figure}
\includegraphics[width=0.85\textwidth]{figures/gp_regression_example.png}
\end{figure}
\end{column}
\end{columns}
\end{frame}


\begin{frame}{Gaussian processes: pros \& cons}

\begin{columns}[t]
\begin{column}{0.5\textwidth}
\begin{myblock}[valign=center]{Pros}
\setlength\leftmargini{0pt}
\begin{itemize}
\item Non-intrusive (can leave legacy code untouched)
\item Easy to use
\item Flexible (non-parametric)
\item Error bands for free
\item $\cdots$
\end{itemize}
\end{myblock}
\end{column}
\begin{column}{0.5\textwidth}
\begin{myblock}[valign=center]{Cons}
\setlength\leftmargini{0pt}
\begin{itemize}
\item Not great at extrapolating
\item Choosing a kernel --- not always straightforward
\item Numerical instabilities can arise
\item Parameter space can be large!
\item Does not necessarily take advantage of structure of the system (more on this later...)
\end{itemize}
\end{myblock}
\end{column}
\end{columns}
\end{frame}


\begin{frame}[t]{Example: GPs for the Schr\"odinger equation}
\begin{columns}[t]
\begin{column}{0.5\textwidth}
\vspace{-0.5cm}%
\begin{align*}
    V(\lecs) = V_{h.o.}(\omega) + \sum \lec_i \exp[-(r/b_i)^2]
\end{align*}
for fixed $\{b_i\} = \{0.5, 2, 4\}$.
\vspace{-0.1cm}
\begin{figure}
\includegraphics[width=0.9\textwidth]{figures/wave_functions_efficient_basis.png}
\end{figure}
\end{column}
\begin{column}{0.5\textwidth}
Fit separate GP to $E$ and radius $R=\E[r]$.
\begin{figure}
\includegraphics[width=0.9\textwidth]{figures/perturbed_oscillator_ground_state_energy_residuals_gp_only.png}
\includegraphics[width=0.9\textwidth]{figures/perturbed_oscillator_ground_state_radius_residuals_gp_only.png}
\end{figure}
\end{column}
\end{columns}
\end{frame}


\begin{frame}{Ritz subspace method: the basics}
Instead of solving the Schr\"odinger eq.\ (an eigenvalue problem for bound states)
\begin{align*}
    H(\lecs) \ket*{\psi(\lecs)} & = E(\lecs) \ket*{\psi(\lecs)}
\end{align*}
write down a \alert{trial wave function} as a linear combination
\begin{align*}
    \ket*{\subspace{\psi}} = \sum_i \beta_i \ket*{\psi_i}
\end{align*}
and use a \alert{variational method} to determine the best $\beta_i$:

\begin{myblock}[valign=center]{Bound state variational method}
Minimize $\mel*{\subspace{\psi}}{H(\lecs)}{\subspace{\psi}}$ such that $\braket*{\subspace{\psi}}{\subspace{\psi}} = 1$.
\end{myblock}

The problem has then been reduced from determining an infinite-dimensional (or, at least large) $\ket*{\psi(\lecs)}$ to determining a couple coefficients $\beta_i$.
\end{frame}



\begin{frame}[plain, fragile, t]{Ritz subspace method for eigenvalue problems (\alert{\href{https://arxiv.org/abs/2104.04441}{arXiv:2104.04441}})}

\begin{columns}
\begin{column}{0.5\textwidth}

\alert{Problem:} $H(\lecs)$ is $N \times N$ and $N \gg 1$
\vspace{-0.1cm}%
\begin{align}
  H(\lecs) \ket*{\psi(\lecs)} & = E(\lecs) \ket*{\psi(\lecs)} \tag{17} \\
  \text{such that}~~~H(\lecs) & = H_0 + \sum \lec_j H_j \notag
\end{align}
\vspace{-0.1cm}%
\alert{Solution:} Choose a basis with $N_b \ll N$:
\begin{align}
    X & \equiv
    \begin{pmatrix}
        \kern.2em\vline\kern.2em & \kern.2em\vline\kern.2em &  & \kern.2em\vline\kern.2em \\
        \ket*{\psi_1} & \ket*{\psi_2} &  \cdots & \ket*{\psi_{N_{b}}} \\
        \kern.2em\vline\kern.2em & \kern.2em\vline\kern.2em & & \kern.2em\vline\kern.2em
    \end{pmatrix} \tag{18} \\
   \subspace{H}(\lecs) & = X^\dagger H(\lecs) X, ~~~~~~ \normmat = X^\dagger X \tag{19}
\end{align}
\vspace{-0.1cm}%
Finally, solve smaller problem:
\begin{align} 
  \subspace{H}(\lecs) \beta(\lecs) = \subspace{E}(\lecs) \normmat \beta(\lecs) \tag{20}
\end{align}
\end{column}
\begin{column}{0.58\textwidth}
\begin{python}[]
def setup_projections(self, X):
    # Project matrices once
    H-*0*-_sub = X.T @ self.H-*0*- @ X # const.
    H-*1*-_sub = X.T @ self.H-*1*- @ X # linear
    # Store for later
    self.X = X; self.N = X.T @ X
    self.H-*0*-_sub = H-*0*-_sub
    self.H-*1*-_sub = H-*1*-_sub
    return self

def solve_subspace(self, a):
    H = self.H-*0*-_sub + self.H-*1*-_sub @ a
    from scipy.linalg import eigh
    E, beta = eigh(H, self.N)
    return E[0], beta[:, 0] # g.s.
\end{python}
\end{column}
\end{columns}
\end{frame}



\begin{frame}[plain, fragile, t]{Ritz subspace method for eigenvalue problems (\alert{\href{https://arxiv.org/abs/2104.04441}{arXiv:2104.04441}})}

\begin{columns}[t]
\begin{column}{0.5\textwidth}

What is gained?
\begin{align}
  H(\lecs) \ket*{\psi(\lecs)} & = E(\lecs) \ket*{\psi(\lecs)} \tag{17} \\
  \text{v.s.} & \notag\\
  \subspace{H}(\lecs) \beta(\lecs) & = \subspace{E}(\lecs) \normmat \beta(\lecs) \tag{20}
\end{align}
Eq.~(20) is much smaller, and
\begin{align}
    E(\lecs) & \approx \subspace{E}(\lecs) & \ket*{\psi(\lecs)} & \approx X\beta(\lecs) \notag
\end{align}
\alert{Emulator} for both $E(\lecs)$ \alert{and} $\ket*{\psi(\lecs)}$! Thus:
\begin{align}
 \ev*{\hat O(\lecs)} & = \!\mel*{\psi(\lecs)}{\hat O(\lecs)}{\psi(\lecs)} \notag \\
 & \approx \beta(\lecs)^\dagger [X^\dagger \hat O(\lecs) X ] \beta(\lecs) \tag{21}
\end{align}
\end{column}
\begin{column}{0.6\textwidth}
\begin{python}[]
def solve_subspace(self, a):
    H = self.H-*0*-_sub + self.H-*1*-_sub @ a
    from scipy.linalg import eigh
    E, beta = eigh(H, self.N)
    # Get ground states:
    return E[0], beta[:, 0]

def predict(self, a):
    E, beta = self.solve_subspace(a)
    psi = self.X @ beta
    return E, psi

def expectation_value(self, a):
    op = self.op-*0*-_sub + self.op-*1*-_sub @ a
    E, beta = self.solve_subspace(a)
    return beta.T @ op @ beta
\end{python}
\end{column}
\end{columns}
\end{frame}

\begin{frame}{Subspace methods: pros \& cons}

\begin{columns}[t]
\begin{column}{0.5\textwidth}
\begin{myblock}[valign=center]{Pros}
\setlength\leftmargini{0pt}
\begin{itemize}
\item Can radically reduce the size of the eigenvector problem
\item If the basis is well chosen, one can get very accurate results
\item Can emulate both the energy and the wave function
\item Gets emulator for downstream observables $\ev*{\hat O(\lecs)}$ for free
\end{itemize}
\end{myblock}
\end{column}
\begin{column}{0.5\textwidth}
\begin{myblock}[valign=center]{Cons}
\setlength\leftmargini{0pt}
\begin{itemize}
\item Intrusive (Requires writing \alert{new} solver code, but it fits in these slides!)
\item Not clear how to choose the basis $\{\ket*{\psi_i}\}$. (\alert{Will fix right now.})
\item No free uncertainty quantification
\item Emulating excited states will require enlarging basis
\end{itemize}
\end{myblock}
\end{column}
\end{columns}
\end{frame}


\begin{frame}{A comparison: GPs vs Ritz (NCSM) subspace emulators}
\begin{columns}[t]
\begin{column}{0.5\textwidth}
Use 6 lowest oscillator states as basis.
\begin{figure}
\includegraphics[width=\textwidth]{figures/perturbed_oscillator_ncsm.png}
\includegraphics[width=\textwidth]{figures/perturbed_oscillator_ground_state_wave_function_residuals_no_ec.png}
\end{figure}
\end{column}
\begin{column}{0.5\textwidth}
Fit a separate GP to $E$ and the radius $R$.
\begin{figure}
\includegraphics[width=\textwidth]{figures/perturbed_oscillator_ground_state_energy_residuals_no_ec.png}
\includegraphics[width=\textwidth]{figures/perturbed_oscillator_ground_state_radius_residuals_no_ec.png}
\end{figure}
\end{column}
\end{columns}
\end{frame}


\section{Efficient Subspace Emulators\\
aka Eigenvector Continuation\\
aka Reduced Basis Method
}


\begin{frame}[t]{Efficient subspace emulators for bound states: the basics}
% \vspace{-1cm}%
% Remember: Write down a \alert{trial wave function} as a linear combination
% \begin{align*}
%     \text{Remember, write down a \alert{trial wave function}:}~~~~\ket*{\subspace{\psi}} = \sum \beta_i \ket*{\psi_i}
% \end{align*}
Remember, write down a \alert{trial wave function}: $\ket*{\subspace{\psi}} = \sum \beta_i \ket*{\psi_i}$

\begin{myblock}[valign=center]{Bound state variational method}
Minimize $\mel*{\subspace{\psi}}{H(\lecs)}{\subspace{\psi}}$ such that $\braket*{\subspace{\psi}}{\subspace{\psi}} = 1$.
\end{myblock}%
\vspace{-0.2cm}%
But how to choose the basis $\{ \ket*{\psi_i} \}$? For parameter-dependent problems:

\begin{myblock}[valign=center]{Efficient Subspace Emulation}
The insight: Use exact solutions $\ket*{\psi(\lecs_i)}$ at a set of training parameters $\{\lecs_i\}$ as the basis for the variational calculation.
\tcblower
The intuition: As the $\lecs$ are varied, the eigenvectors only trace a small subspace compared to the full Hilbert space. Using exact solutions thus automatically finds an \alert{incredibly} effective basis for subsequent emulation.
\end{myblock}

\end{frame}


\begin{frame}[plain, fragile, t]{Efficient subspace emulators for bound states: the code}

\begin{columns}[t]
\begin{column}{0.5\textwidth}
\vspace{-0.5cm}%
\begin{align*}
    V(\lecs) = V_{h.o.}(\omega) + \sum \lec_i \exp[-(r/b_i)^2]
\end{align*}
for fixed $\{b_i\} = \{0.5, 2, 4\}$.
\vspace{-0.1cm}
\begin{figure}
\includegraphics[width=\textwidth]{figures/wave_functions_efficient_basis.png}
\end{figure}
\end{column}
\begin{column}{0.57\textwidth}
\begin{python}[]
import numpy as np

def fit(self, a_train):
    # Create subspace from exact $\ket{\psi(\lecs)}$
    X = []
    for a in a_train:
        E, psi = self.solve_exact(a)
        X.append(psi)
    # Stack them as columns
    X = np.stack(X, axis=1)

    # The same function as before:
    self.setup_projections(X)
    # Store the training points
    self.a_train = a_train
    return self
\end{python}
\end{column}
\end{columns}
\end{frame}


\begin{frame}[t]{Another comparison: GPs vs NCSM vs efficient emulators}
\begin{columns}
\begin{column}{0.5\textwidth}
\onslide<1>{Inefficient! Most training wave functions do not look like the emulated states\\}
\only<1>{\includegraphics[width=\textwidth]{figures/perturbed_oscillator_ncsm.png}}
\only<2>{\includegraphics[width=\textwidth]{figures/perturbed_oscillator_ground_state_energy_residuals.png}}

\end{column}
\begin{column}{0.5\textwidth}
\onslide<1>{Much smaller basis span, but much more efficient!\\}
\only<1>{\includegraphics[width=\textwidth]{figures/perturbed_oscillator_efficient.png}}
\only<2>{\includegraphics[width=\textwidth]{figures/perturbed_oscillator_ground_state_radius_residuals.png}}
\end{column}
\end{columns}

\only<2>{Comparing GP, NCSM, and the efficient emulators: the efficient emulator wins!}
\begin{center}\vspace{-1cm}
\only<1>{\includegraphics[width=0.5\textwidth]{figures/perturbed_oscillator_ground_state_wave_function_residuals.png}}
\end{center}
\end{frame}


\begin{frame}{Bound state emulators in the wild (\alert{K\"onig \emph{et al.} \href{https://arxiv.org/abs/1909.08446}{arXiv:1909.08446}})}
\begin{columns}
\begin{column}{0.5\textwidth}
\begin{figure}
\includegraphics[width=\textwidth]{figures/konig_interpolation_energy.png}
\end{figure}
\end{column}
\begin{column}{0.5\textwidth}
\begin{figure}
\includegraphics[width=\textwidth]{figures/konig_extrapolation_energy.png}
\end{figure}
\end{column}
\end{columns}
Eigenvector continuation (the efficient subspace emulator) beats both polynomials and GPs on interpolation and extrapolation in parameter space.

Observable: $^{4}$He g.s.\ energy. Uses 12 training points with 3 parameters.
\end{frame}


\begin{frame}{Bound state emulators in the wild: inverse problem (\alert{\href{https://arxiv.org/abs/2104.04441}{arXiv:2104.04441}})}
\begin{columns}[t]
\begin{column}{0.5\textwidth}
\begin{figure}
\only<1>{\includegraphics[width=0.8\textwidth]{figures/nn_training_pts_with_stdv_titles.png}}
\end{figure}
\only<1>{Sampling 15 different parameters (some not shown). 50 training points.}
\only<2>{Able to rapidly perform sampling on laptop in minutes, rather than on supercomputer for hours}
\end{column}
\begin{column}{0.5\textwidth}
\begin{figure}
\only<1>{\includegraphics[width=0.8\textwidth]{figures/validation_residuals.png}}
\only<2>{\includegraphics[width=0.9\textwidth]{figures/posterior_samples_cd_ce_clean.png}}
\end{figure}
\only<1>{Negligible residuals on validation data}
\end{column}
\end{columns}
\end{frame}


\section{Beyond Bound States}

\begin{frame}[t]{Efficient trial scattering wave functions: the basics (\alert{\href{https://arxiv.org/abs/2007.03635}{arXiv:2007.03635}})}
Again, write down a \alert{trial wave function}: $\ket*{\subspace{\psi}} = \sum \beta_i \ket*{\psi_i}$.

For scattering, it is convenient to work with the radial wave function $u_\ell(r)$ for partial wave $\ell$, which in asymptotic form is
\begin{align*}
    u_\ell(r) \xrightarrow[r\to\infty]{} \sin(pr - \frac{1}{2}\ell\pi) + K_\ell \cos(pr - \frac{1}{2}\ell\pi)
\end{align*}

\begin{myblock}[valign=center]{Kohn variational principle (KVP)}
Minimize $\mathcal{K}_{KVP}[\subspace{\psi}_\ell] = K_\ell - \mel*{\subspace{\psi}_\ell}{H(\lecs) - E}{\subspace{\psi}_\ell}$ such that $\braket*{\subspace{\psi}_\ell}{\subspace{\psi}_\ell} = 1$.
\end{myblock}%

Again, use exact $u_\ell(r)$ at training points $\{\lecs_i\}$ as the basis.

It is a linear problem, can solve for $\beta$ analytically and quickly.
\end{frame}

\begin{frame}{Trial scattering wave functions in the wild (\alert{\href{https://arxiv.org/abs/2007.03635}{arXiv:2007.03635}})}

\begin{columns}
\begin{column}{0.5\textwidth}
\only<2>{Can accurately emulate with \alert{Coulomb}}
\only<3>{Also works with \alert{optical} potentials}
\end{column}
\begin{column}{0.5\textwidth}

\begin{figure}
\only<2>{\includegraphics[width=\textwidth]{figures/furnstahl_p_alpha_residuals.png}}
\only<3>{\includegraphics[width=\textwidth]{figures/furnstahl_optical_residuals.png}}
\end{figure}
\end{column}
\end{columns}
\only<1>{Can emulate entire scattering wave function and its phase shifts}
\only<1>{%
\begin{figure}%
\includegraphics[width=0.9\textwidth]{figures/emulated_scattering_wave_functions.png}%
\end{figure}%
}
\end{frame}


\begin{frame}{Efficient trial $K$ or $T$ matrices (\alert{\href{https://arxiv.org/abs/2106.15608}{arXiv:2106.15608}})}

Rather than solve the Schr{\"o}dinger equation, use the Lippmann--Schwinger (LS) equation
\begin{align*}
    K = V + V G_0 K
\end{align*}
Propose \alert{trial $K$ matrix}
\begin{align*}
    \subspace{K} = \sum \beta_i K_i
\end{align*}

\begin{myblock}[valign=center]{Newton variational principle (NVP)}
Minimize $\mathcal{K}_{NVP}[\subspace{K}] = V + V G_0 \subspace{K} + \subspace{K} G_0 V - \subspace{K} G_0 \subspace{K} + \subspace{K} G_0 V G_0 \subspace{K}$ ~~(no constraints!)
\end{myblock}%

As usual, take the \alert{matrix basis} $\{K_i\}$ from exact solutions of the LS equation.

It is a linear problem, can solve for $\beta$ analytically and quickly.

\end{frame}

\begin{frame}{Trial $K$ matrices in the wild (\alert{\href{https://arxiv.org/abs/2106.15608}{arXiv:2106.15608}})}

\begin{columns}
\begin{column}{0.5\textwidth}
\begin{figure}
\only<1>{\includegraphics[width=\textwidth]{figures/minnesota_1S0_phases_with_residuals.png}}
\only<2>{Can accurately \& efficiently emulate \alert{gradients} with respect to parameters}
\only<3>{Can easily handle the \alert{Coulomb} interaction}
\only<5>{%
\begin{align*}%
    \sigma_{\text{tot}}(q) = - \frac{\pi}{2q^2} \sum_{j=0}^{j_{\text{max}}} (2j+1) \Re{ \Tr [S_j(q)-\mathds{1}] }%
\end{align*}%
Multiple emulators across partial waves can be combined to emulate scattering observables. Over \alert{300x} improvement in CPU time.%
}
\end{figure}
\end{column}
\begin{column}{0.5\textwidth}
\begin{figure}
\only<1>{\includegraphics[width=\textwidth]{figures/minnesota_1S0_extrapolation_K.png}}
\only<2>{\includegraphics[width=\textwidth]{figures/minnesota_1S0_K_grads_with_residuals.png}}
\only<3>{\includegraphics[width=\textwidth]{figures/coulomb_p_alpha_1S0_phases_and_residuals.png}}
\only<5>{\includegraphics[width=\textwidth]{figures/np_total_cross_section_with_errors.png}}
\end{figure}
\end{column}
\end{columns}
\only<1>{Can \alert{extrapolate} very far from support of data, even across singularities}
\only<4>{Coupled channels get major speedups with emulation}
\only<4>{\includegraphics[width=\textwidth]{figures/np_3S1_3D1_coupled_K_matrix_with_residual.png}}
\end{frame}


\begin{frame}{Density functional theory?! (In progress)}
% \begin{columns}
% \begin{column}{0.5\textwidth}

% \end{column}

% \begin{column}{0.5\textwidth}
% \begin{figure}

% \end{figure}
% \end{column}
% \end{columns}

\only<1>{
Two reasonable approaches
\begin{enumerate}
\item The Kohn-Sham formalism requires self-consistently solving Schr{\"o}dinger equations for orbitals. One could emulate this step.
\item The ground state energy is minimized as a functional of the density. Just write down a trial density and turn the crank.
\end{enumerate}
I took the latter approach
\begin{align*}
\subspace{\rho} = \sum \beta_i \rho_i
\end{align*}
and minimized
\begin{align*}
E[\subspace{\rho}] = g\sum \epsilon - a \int \dd[3]{x} [\subspace{\rho}(x)]^2 - b \int \dd[3]{x} [\subspace{\rho}(x)]^{7/3} - c \int \dd[3]{x} [\subspace{\rho}(x)]^{8/3}
\end{align*}
Non-linear means there is no ``nice'' solution for $\beta$. Use an optimizer.
}

\only<2>{
\begin{figure}
\includegraphics[width=0.45\textwidth]{figures/dft_nlo_example.png}
\end{figure}
}

% \begin{columns}
% \begin{column}{0.5\textwidth}

% \end{column}
% \begin{column}{0.5\textwidth}

% \end{column}
% \end{columns}
\end{frame}


\begin{frame}{Uncertainty quantification for subspace emulators (\alert{\href{http://rave.ohiolink.edu/etdc/view?acc_num=osu1587114253866152}{My thesis}})}

\begin{columns}[t]
\begin{column}{0.5\textwidth}
\begin{itemize}
\item But wait! What about emulator uncertainty?
\item Not much published to date (that I know of)
\item I made a proposal in my thesis
\begin{align*}
    \ket*{\psi(\lecs)} & = \ket*{\subspace{\psi}(\lecs)} + \ket*{\epsilon(\lecs)} \\
    \ket*{\epsilon(\lecs)} & \sim \mathcal{GP}[0, W \kappa(\lecs, \lecs'; \boldsymbol{\theta})]
\end{align*}
\item Train with leave-$k$-out CV only on basis wave functions
\item Uncertainties on downstream observables are pure predictions
\end{itemize}
\end{column}
\begin{column}{0.5\textwidth}
\begin{figure}
\includegraphics[width=\textwidth]{figures/residuals_and_errors_triton_halflife.png}
% \includegraphics[width=\textwidth]{figures/residuals_and_errors_He4_rt_r2.png}
\end{figure}
\end{column}
\end{columns}
\end{frame}


\section{Concluding Remarks}

\begin{frame}{Conclusions}

\begin{columns}[t]
\begin{column}{0.4\textwidth}
\begin{myblock}[valign=center]{BAND}
The goal of BAND is to translate novel statistical methods of UQ into
software tools that address prominent current problems in nuclear physics.
\end{myblock}
Subspace emulation could play a key role in this effort
\end{column}
\begin{column}{0.6\textwidth}
\begin{myblock}[valign=center]{Benefits}
\setlength\leftmargini{0pt}
\begin{itemize}
\item Can radically reduce the size of the problem (not just \alert{eigenvectors}!)
\item An extremely effective basis can be chosen \alert{automatically}
\item Gets emulator for downstream observables $\ev*{\hat O(\lecs)}$ \alert{for free}
\end{itemize}
\end{myblock}
\begin{itemize}
\item Promising directions: heavy systems, beyond eigenvalues, DFT, etc.
\item Error bands would be great!
\end{itemize}
\end{column}
\end{columns}
\end{frame}



\appendix

\begin{frame}[standout,t]

\vspace{.2in}

Thank you!
\noindent\rule{\textwidth}{1pt}

% The BUQEYE collaboration
% \begin{center}
% % \href{https://arxiv.org/abs/1808.08211}{\alert{arXiv:1808.08211}}
% \href{https://arxiv.org/abs/1904.10581}{\alert{arXiv:1904.10581}}
% \hspace{.3in}
% \href{https://arxiv.org/abs/1704.03308}{\alert{arXiv:1704.03308}}
% \\
% \href{https://buqeye.github.io/}{\alert{buqeye.github.io}}
% \hspace{.3in}
% \href{https://buqeye.github.io/gsum/}{\alert{buqeye.github.io/gsum}}
% \end{center}

\begin{columns}
\begin{column}{0.5\textwidth}
\begin{center}
\href{https://buqeye.github.io/}{\alert{buqeye.github.io}}
\end{center}
\end{column}
\end{columns}
\end{frame}

\end{document}
