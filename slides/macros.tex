% !TEX root = CH_to_EKM_revisited_v4.tex

% Define \newsubcommand:
% look whether the following character is _ and, if so, it will merge the subscripts.
\makeatletter
\newcommand\newsubcommand[3]{\newcommand#1{#2\sc@sub{#3}}}
\def\sc@sub#1{\def\sc@thesub{#1}\@ifnextchar_{\sc@mergesubs}{_{\sc@thesub}}}
\def\sc@mergesubs_#1{_{\sc@thesub#1}}
\makeatother


\DeclareDocumentCommand{\includevcenteredgraphics}{ O{} m }{%
    \vcenter{\hbox{\includegraphics[#1]{#2}}}
}

\newcommand{\subspace}[1]{ \ensuremath{\widetilde{#1}} }
\newcommand{\normmat}{\mathcal{N}}

\newcommand{\lec}{a}
\newcommand{\lecvec}{\vec{\lec}}
\newcommand{\lecs}{\vec{\lec}}
\newcommand{\Emax}{E_\textup{max}}
\newcommand{\kmax}{k_\textup{max}}

%\newcommand{\NNLO}{N$^2$LO}
%\newcommand{\NNNLO}{N$^3$LO}
\newcommand{\LO}{\ensuremath{{\rm LO}}}
\newcommand{\NLO}{\ensuremath{{\rm NLO}}}
\newcommand{\NNLO}{\ensuremath{{\rm N}{}^2{\rm LO}}}
\newcommand{\NNNLO}{\ensuremath{{\rm N}{}^3{\rm LO}}}
\newcommand{\NNNNLO}{\ensuremath{{\rm N}{}^4{\rm LO}}}
\newcommand{\NkLO}[1]{\ensuremath{\mathrm{N}^{#1}\mathrm{LO}}}

\DeclareMathOperator{\E}{\mathbb{E}}


\newcommand{\qgeo}[1]{q_k^{(#1)}}
\newcommand{\neff}{\ensuremath{N_{\text{eff}}}}
\newcommand{\data}{\mathcal{D}}
% \newcommand{\hyperparam}{\boldsymbol{\theta}}
\newcommand{\param}{\boldsymbol{\theta}}
\newcommand{\len}{\ell}
\newcommand{\lenvec}{\boldsymbol{\ell}}
\newcommand{\lenset}{\ensuremath{L}}
\newcommand{\lenvecset}{\ensuremath{\mathbf{L}}}
\DeclareMathOperator{\GP}{\mathcal{GP}}
\DeclareMathOperator{\TP}{\mathcal{TP}}

% Neutron-proton
\newcommand{\npr}{\ensuremath{np}}

\newcommand{\chiEFT}{\ensuremath{\chi}\text{EFT}}

\newcommand{\boldvec}[1]{\bm{#1}}
%\newcommand{\boldvec}[1]{\mathbf{#1}}


\newcommand{\abar}{\bar a}

\newsubcommand{\ckvec}{\mathbf{c}}{k}
% \newcommand{\ckvecsq}{\mathbf{c}_k^{2}}
\newsubcommand{\bkvec}{\mathbf{b}}{k}
% \newcommand{\bkvecsq}{\mathbf{b}_k^{2}}
% \newcommand{\kinparvec}{\boldvec{\alpha}}
\newcommand{\kinparvec}{x}
\newcommand{\kinparvecset}{\vb{X}}
\newcommand{\kinparvecmag}{x}
\newsubcommand{\ckvecset}{\mathbf{C}}{k}
\newcommand{\cbarset}{\bar{C}}
\newcommand{\cset}{C}
\newcommand{\capprox}{\ensuremath{c'}}
\newcommand{\capproxset}{\ensuremath{C'}}
\newsubcommand{\ckvecapprox}{\mathbf{c}'}{k}
\newsubcommand{\ckvecapproxset}{\mathbf{C}'}{k}
\newcommand{\bset}{B}
\newcommand{\bapprox}{\ensuremath{b'}}
\newcommand{\bapproxset}{\ensuremath{B'}}
\newsubcommand{\bkvecapprox}{\mathbf{b}'}{k}
\newsubcommand{\bkvecset}{\mathbf{B}}{k}
\newsubcommand{\bkvecapproxset}{\mathbf{B}'}{k}
\newcommand{\genobs}{y}
\newsubcommand{\genobsth}{y}{\textup{th}}
\newsubcommand{\genobsexp}{y}{\textup{exp}}
\newsubcommand{\genobsvec}{\boldsymbol{y}}{\!k}

\newsubcommand{\akvec}{\mathbf{a}}{k}
\newcommand{\aset}{A}
\newcommand{\aapprox}{\ensuremath{a'}}
\newcommand{\aapproxset}{\ensuremath{A'}}
\newsubcommand{\akvecapprox}{\mathbf{a}'}{k}
\newsubcommand{\akvecset}{\mathbf{A}}{k}
\newsubcommand{\akvecapproxset}{\mathbf{A}'}{k}


\newcommand{\Covfunc}{\ensuremath{k}}
\newcommand{\Covmat}{\vb{K}}
\newcommand{\Corrmat}{\vb{R}}
\newcommand{\Covmattest}{\boldsymbol{\Sigma}}


%%%%%%%%%%%%%%%%%%%%%%%%

%%%%%%%%%%%%%%%%%%%
% The \pr command
%%%%%%%%%%%%%%%%%%%

\let\Pr\undefined{}  % Remove the definition from the Physics package

% \newcommand{\pr}{\text{pr}} % Old
\DeclareMathOperator{\pr}{pr} % Good, but want to handle sizing and | spacing?
% \DeclareDocumentCommand\pr{}{\opbraces{\prob}} % Using opbraces from physics package to resize delimeters
\newcommand{\given}{\,|\,}  % Use for | in \pr
\DeclareMathOperator{\prior}{\pi}


\newcommand{\cbar}{\bar{c}}


\newcommand{\Q}{\ensuremath{Q}}
\newcommand{\Qapprox}{\ensuremath{Q'}}
\newcommand{\Qmax}{\Q_{\rm max}}
\newcommand{\smallscale}{\ensuremath{\mathcal{Q}}}
\newcommand{\smallscaleapprox}{\ensuremath{\mathcal{Q}'}}

\newcommand{\Elab}{E_{\rm lab}}


\newcommand{\normal}{\mathcal{N}}
\newcommand{\IG}{\textup{IG}}


\newcommand{\transpose}[1]{#1^{\intercal}}
% \newcommand{\trans}{\intercal}



\newcommand{\genobsref}{\ensuremath{y_{\textup{ref}}}}
\newcommand{\Xref}{\genobsref}


% Differential taken from Physics package
% Uses smart spacing for a nice look. Requires xparse.
% https://www.ctan.org/pkg/physics?lang=en

% First, the basics:
\def\diffd{\mathrm{d}}  % Upright differentials
% \def\diffd{d}  % Italic differentials

% Now add spacing:
% Derivatives
\DeclareDocumentCommand\differential{ o g d() }{ % Differential 'd'
    % o: optional n for nth differential
    % g: optional argument for readability and to control spacing
    % d: long-form as in d(cos x)
    \IfNoValueTF{#2}{
        \IfNoValueTF{#3}
            {\diffd\IfNoValueTF{#1}{}{^{#1}}}
            {\mathinner{\diffd\IfNoValueTF{#1}{}{^{#1}}\argopen(#3\argclose)}}
        }
        {\mathinner{\diffd\IfNoValueTF{#1}{}{^{#1}}#2} \IfNoValueTF{#3}{}{(#3)}}
    }
\DeclareDocumentCommand\dd{}{\differential} % Shorthand for \differential


% Path Derivative:
\newcommand{\pathd}{\mathcal{D}}  % differential symbol for path integrals

% Now add relevant spacing and options
% Inspired by differential definition from `Physics' package at
% https://www.ctan.org/tex-archive/macros/latex/contrib/physics?lang=en
\DeclareDocumentCommand\pathdifferential{ o g d() }{ % Path 'D'
    % o: optional n for nth differential
    % g: optional argument for readability and to control spacing
    % d: long-form as in d(cos x)
    \IfNoValueTF{#2}{
        \IfNoValueTF{#3}
            {\pathd\IfNoValueTF{#1}{}{^{#1}}}
            {\mathinner{\pathd\IfNoValueTF{#1}{}{^{#1}}\argopen(#3\argclose)}}
        }
        {\mathinner{\pathd\IfNoValueTF{#1}{}{^{#1}}#2} \IfNoValueTF{#3}{}{(#3)}}
    }

\DeclareDocumentCommand\DD{}{\pathdifferential} % Shorthand for \pathdifferential

% Observable notation

\makeatletter
\def\dsigma{\@ifstar\@@dsigmawithstar\@dsigma}
\def\@dsigma{\frac{\dd\sigma}{\dd\Omega}}  % \dsigma for upright frac
\def\@@dsigmawithstar{\dd\sigma/\dd\Omega} % \dsigma* for inline frac
\makeatother
